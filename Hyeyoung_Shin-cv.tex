%_______________________________________________
% @brief    LaTeX2e Resume for Hyeyoung Shin
\documentclass[margin,line]{resume}
\usepackage{url}
\usepackage[shortlabels]{enumitem}
%\usepackage{enumitem}

\usepackage{graphicx}

\usepackage[usenames,dvipsnames]{color}
\definecolor{light-gray}{gray}{0.70}
\definecolor{gray}{gray}{0.50}
\definecolor{dark-gray}{gray}{0.30}

\usepackage[colorlinks=true]{hyperref}

\hypersetup{
     citecolor    = gray,
     urlcolor=Blue
     %% linkcolor=Emerald
}

\newcommand\mysmallskip{4pt}
    \newcommand\mymedskip{10pt}
    \newcommand\mybigskip{14pt}


\usepackage{ifthen}
\usepackage[LabelsAligned]{currvita}     
\begin{document}

\newcommand{\Hawaii}{Hawai\kern.05em`\kern.05em\relax i}
\newcommand{\Manoa}{M\=anoa}
\name{{\Large Hyeyoung Shin} \hskip127mm {\large\sc Resum\'e}} %{\large\sc Curriculum Vit\ae}}

\begin{resume}

    %___________________________________________
    % Contact Information
    \section{\mysidestyle Contact\\Information}
     91 Camino Bosque              \hfill    \href{https://www.hyeyoung.org}{www.hyeyoung.org}\\
     Boulder, CO 80302             \hfill    \href{mailto:hyeyoungshinw@gmail.com}{hyeyoungshinw@gmail.com}

    %__________________________________________________________________________________
    % Research Interests
    \section{\mysidestyle Professional Identification}
    Intermediate-level Research and Development Engineer with two years of post-masters experience in academia and two years in industry, passionate about solving complex problems with logic and data-driven solutions.
    

    % Avid and curious developer with a passion for performance and elegance. Love functional programming and developing algorithms 
    % that solve complex problems. Three and a half years of post-masters experience developing large scale software systems and open 
    % source libraries.
    
    
    %________________________________________________________________________________
    % Education
    \section{\mysidestyle Education}

    \textbf{Master of Science in Computer Science} \hfill Northeastern University 2019\\
    Project: \href{https://github.com/hyeyoungshin/popl19src/blob/master/popl19src.pdf}{Secure Compilation}\hfill Advisor: \href{https://www.khoury.northeastern.edu/home/amal/}{Amal Ahmed}\\[\mymedskip]
    % Advisor: \href{http://www.ccs.neu.edu/home/amal/}{Prof.~Amal Ahmed}
    %
    % \textbf{University of \Hawaii, \Manoa}. Course work in mathematics
    % and computer science \hfill 2016--2017\\[\mymedskip]  %\hfill Ames, IA\\
    %% {\small Graduate: logic, recursion theory; undergraduate: concurrent programming, topology} \\[\mymedskip]
    %
    % \textbf{Iowa State University}. Course work in mathematics and computer science \hfill 2014--2016\\[\mymedskip]
    %% {\small Graduate: Programming languages(Coq), formal methods,  computability;\\
    %% Undergraduate: OOP, data structures, algorithms, abstract algebra, intro to proofs, calculus} \\[\mymedskip]
    %
    \textbf{Bachelor of Arts in English Language and Literature}  \hfill Kyeongpook National University 2009  %\hfill Daegu, South Korea\\
    

    %_______________________________________________________________
    % Work Experience
    \section{\mysidestyle Professional\\Experience}
    \textbf{Library Engineer} \hfill {\href{https://relational.ai/}{RelationalAI}} 2022--2024\\
    {\small Developed an advanced graph analytics library for the knowledge graph coprocessor.}\\[\mymedskip]
    %
    \textbf{Research and Development} \hfill {\href{https://prl-prg.github.io/}{Programming Research Laboratory}}, Czech Technical University 2019--2021\\
    {\small Contributed to an \href{https://dl.acm.org/doi/abs/10.1145/3567512.3567530}{open-source, data-driven fuzzing tool for R}, enhancing software reliability and security.}\\[\mymedskip]
    % 
    %%
    \textbf{Teaching Assistant} \hfill {Czech Technical University} Fall 2020\\
    {\small Educated international students in Object-Oriented Programming (OOP) design using Scala.}\\[\mymedskip]
    %%
    \textbf{Research Assistant} \hfill {\href{https://prl.ccs.neu.edu/}{Programming Research Laboratory}, Northeastern University} Fall 2018\\
    {\small Worked on secure compilation projects, focusing on enhancing the safety and security of software systems through innovative compilation verification techniques.}\\[\mymedskip]
    %%
    \textbf{Teaching Assistant} \hfill {Iowa State University} Fall 2015\\
    {\small Led weekly recitations for a Java data structures course, providing hands-on learning and support to students.}
    

    %_______________________________________________________________
    % Publications and Projects
    \section{\mysidestyle Publications \\and \\Projects}
    \begin{itemize}
        \item \href{https://docs.relational.ai/preview/graph-analytics/overview}{\textbf{Graph Analytics Library}}: Played a key role in developing a comprehensive graph analytics toolkit at \href{https://relational.ai/}{RelationalAI} for the knowledge graph coprocessor, transforming raw data into actionable insights and enabling data-driven decision-making.
        \item \href{https://github.com/PRL-PRG/sle22-signatr-artifact}{\textbf{Data-Driven Fuzzing Tool for R}}: Developed at the Programming Research Laboratory, Czech Technical University, improving the robustness of R applications and contributing to the open-source community.
        \item \href{https://www.khoury.northeastern.edu/research_projects/shf-small-secure-compilation-of-advanced-languages/}{\textbf{Secure Compilation Research}}: Participated in a project aimed at enhancing software security through advanced compilation verification techniques, with an \href{https://github.com/hyeyoungshin/popl19src/blob/master/popl19src.pdf}{abstract} submitted to a \href{https://popl19.sigplan.org/track/POPL-2019-Student-Research-Competition?}{student research competition at Principles of Programming Languages 2019}.
    \end{itemize}

    
    %______________________________________________________________________________________
    \section{\mysidestyle Technical\\Skills}
    %
    \textbf{Programming Languages}: R, Scala, Java, Python\\[\mymedskip]
    \textbf{DevOps}: Docker, AWS, Azure\\[\mymedskip]
    \textbf{Tools \& Platforms}: Git, Github, Gitlab, JIRA, Emacs, Visual Studio Code\\[\mymedskip]
    \textbf{Testing}: Unit Testing, Itegration Testing, CI/CD
    

    
    % \textbf{R}.  {\small Implemented an open source tool that infers function type signatures by fuzz testing with data acquired by tracing source code from CRAN, designed a structured database with a query API, conducted exploratory data analysis and visualization  (\href{https://github.com/PRL-PRG/signatr}{source code link}).}\\[\mymedskip]
    %
    % \textbf{Scala}. {\small Developed an API to automatically produce webservers  .}\\[\mymedskip]
    %
    %\textbf{Java}. {\small Wrote a number of business applications including language popularity ranking by collecting and analyzing stackoverflow data.}\\[\mymedskip]
    %
    % \textbf{C/C++}. {\small Designed a dynamic tracer which instruments the execution of R programs with customized hooks in C (\href{https://github.com/PRL-PRG/argtracer}{source code link}).}\\[\mymedskip]
    %\textbf{Racket.} {\small Implemented interpreter generator (\href{https://github.com/hyeyoungshin/hy_eopl}{source code link}) parametrized by representations of environment and closure}\\[\mymedskip]
    %\textbf{SML.} {\small Implemented a compiler that compiles \href{https://www.cs.princeton.edu/~appel/modern/ml/}{Tiger language} to {\small MIPS} assembly}\\[\mymedskip]
    %\textbf{Coq.} {\small Worked through Software Foundations by Pierce et al. Received A in graduate PL course.}\\[\mymedskip]
    %
    % \textbf{Other}. Java, Julia, Python, Scala, Git, Unix, LaTeX



    %______________________________________________________________________________________
%\section{\mysidestyle Work Experience}

%\newcommand\ssitem[5]{\href{#1}{#2} \hfill #3 \\ Topics: #4 \hfill #5}

%\ssitem
%{https://bigcode-prl-prg.github.io/}
%{Big Code: Scalable Analysis of Massive Code Bases}
%{Faculty of Information Technology,\\
%Czech Technical University in Prague}
%{lambda calculus, category theory, univalent type theory in Agda}
%{Nov 2019--Oct 2021}\\[\mymedskip] % 14--18 
%\ssitem
%{https://summer-school.racket-lang.org/2017/}
%{The Racket School of Semantics and Languages}
%{University of Utah}
%{semantics and language design}
%{July 2017}\\[\mymedskip] % 10 July--14 
%\ssitem
%{https://www.cs.uoregon.edu/research/summerschool/summer17}
%{Oregon Programming Languages Summer School}
%{University of Oregon}
%{dependent, gradual, and substructural type systems}
%{June 2017}\\[\mymedskip] % 26 June--8 July
%\ssitem
%{http://www.cs.bham.ac.uk/~pbl/mgs2016/}
%{Midlands Graduate School in Foundations of Computing Science}
%{University of Birmingham}
%{type theory, denotational semantics, category theory}
%{April 2016}\\[\mymedskip] % 11--15 
%\ssitem
%{https://www.cs.uoregon.edu/research/summerschool/summer16/}
%{Oregon Programming Languages Summer School}
%{University of Oregon}
%{type theory, logic, semantics, verification}
%{June 2016}\\[\mymedskip]  % 20 June--2 July 
%\ssitem
%{https://www.coursera.org/learn/progfun1}
%{Functional Programming Principles in Scala}
%{\'{E}cole Polytechnique F\'{e}d\'{e}rale de Lausanne}
%{6-week online course with
%\href{https://www.coursera.org/account/accomplishments/records/SRLRBNFMFW86}
%     {verified certificate}}
%                        {Grade Achieved: 94\%}



    
%%     %_______________________________________________________________    %         Volunteer work
%%     \section{\mysidestyle Volunteer\\Work}

%%     \textbf{InDaegu} \hfill 2011--2012\\
%%     \textsl{Korean-English translator:} 
%%     \href{http://www.in-daegu.com/}{InDaegu newspaper}.
%% %    \href{http://www.in-daegu.com/}{(in-daegu.com)}. 
    
%%         \textbf{Daegu Pockets}  \hfill 2010--2011\\
%%     \textsl{Korean-English translator:}
%%     \href{http://daegupockets.com/}{Daegu Pockets magazine}.
%% %    \href{http://daegupockets.com/}{(daegupockets.com).}
    
%%     \textbf{Daegu International Bodypainting Festival}  \hfill 2010\\
%%     \textsl{Korean-English translator:} DIBF Association.
    
%%     \textbf{Daegu International Musical Festival}   \hfill 2009\\
%%     \textsl{Korean-English translator:} DIMP Association.

    \renewcommand\mymedskip{3pt}




    \section{\mysidestyle Honors} 
    %% \begin{itemize}
    %% \item Scholarship to attend
    %%   \href{https://www.cs.uoregon.edu/research/summerschool/summer16/}{OPLSS},
    %%   Oregon; 20 June--2 July, 2016.
    %% \item Scholarship to attend
    %%   \href{https://www.cis.upenn.edu/~sweirich/icfp-plmw15/}
    %%        {Programming     Languages Mentoring Workshop} at the\\
    %%        20th ACM SIGPLAN
    %%        {\bf \href{http://icfpconference.org/icfp2015/}{ICFP 2015}}
    %%        %% International Conference on Functional Programming, 
    %%        Vancouver, BC; 31 Aug--2 Sep, 2015.
    %% \item 
    %%   Scolarship to atten the
    %%   {\bf \href{http://conf.researchr.org/home/PLMW-2016}{POPL/PLMW 2016}}
    %%   Florida; 19 Jan--22 Jan, 2016.
    %% \end{itemize}
    Northeastern University Graduate Fellowship\hfill 2018\\[\mymedskip]
    \href{https://www.sigplan.org/Conferences/PLMW/}{Programming Languages Mentoring Workshop Scolarship} \hfill 2016, 2017\\[\mymedskip]
    \href{https://www.cs.uoregon.edu/research/summerschool/summer17}{Oregon Programming Languages Summer Schools fellowship} \hfill %% ; 20 June--2 July, 
    2016, 2017
    
    
%     Scholarship to attend
%     \href{https://www.cs.uoregon.edu/research/summerschool/summer16}
%          {Oregon Programming Languages Summer School} \hfill %% ; 20 June--2 July, 
%     Eugene, 2016\\[\mymedskip]
%    Scholarship to attend 
%     \href{http://conf.researchr.org/home/PLMW-2016}{{\small POPL}
%        Programming Languages Mentoring Workshop} \hfill St.~Petersburg, 2016\\[\mymedskip]
%            Scholarship to attend
%           \href{http://icfpconference.org/icfp2015/}{{\small ICFP}}
%           \href{https://www.cis.upenn.edu/~sweirich/icfp-plmw15/}
%                {Programming Languages Mentoring Workshop}
                %% at the\\    20th ACM SIGPLAN
         %% International Conference on Functional Programming, 
%\hfill         Vancouver, %% , BC; 31 Aug--2 Sep, 
%2015


   
    % \section{\mysidestyle References} 
    % 
\newcommand\referee[6]{\textbf{#1}\\{\small #2}\\{\small #3}\\{\small #4}\\{\small #5}\\{\small email: \url{#6}}}


    \begin{tabular}{@{}p{6cm}p{6cm}}
      \referee
          {Jan Vitek}
          {The Khoury College of Computer Sciences}
          {Northeastern University}
          {440 Huntington Ave}
          {Boston, MA 02115}
          {vitekj@me.com}
    \end{tabular}
    \begin{tabular}{@{}p{6cm}p{6cm}}
      \referee
          {Christoph Kirsch}
          {Department of Computer Sciences}
          {University of Salzburg}
          {Jakob-Haringer-Str. 2}
          {5020 Salzburg, Austria}
          {kirscchr@fit.cvut.cz}
    \end{tabular}




    % \newpage
    %%\section{\mysidestyle Talks}
    %% \vspace{-3mm}
    %%  \href{https://github.com/hyeyoungshin/Talks/raw/master/HTT/htt-slides.pdf}
    %%       {Hoare Type Theory}, July 2016.


    %%\begin{center}
    %%RELEVANT COURSEWORK
    %%\end{center}
    %%______________________________________________________________________________________
    %%\section{\mysidestyle Computer Science}
    %%\begin{itemize}
%%    \item Graduate Programming Languages ({\small COMS} 542; grade: A)
%%    \item Graduate Theory of Computing ({\small COMS} 531; grade: A)
%%    \item Graduate Formal Methods ({\small COMS} 512; grade: A)
%%    \item Undergraduate Theory of Computing ({\small COMS} 331; grade: A-)
%%    \item Discrete Computational Structures ({\small COMS 330}; grade: A)
%%    \item Algorithms ({\small COMS} 311; grade: A)
%%    \item Data Structures ({\small COMS} 228; grade: A)
%%    \item Object-oriented Programming ({\small COMS} 227; grade: A)
%%    \end{itemize}

%%    %______________________________________________________________________________________
%%    \section{\mysidestyle Mathematics}
%%    \begin{itemize}
%%    \item Graduate Logic ({\small MATH} 654; grade: tbd)
%%    \item Topology ({\small MATH} 421; grade: tbd)
%%    \item Abstract Algebra ({\small MATH} 301; grade: A)
%%    \item Introduction to Proofs ({\small MATH} 201; grade: A)
%%    \item Calculus I, II ({\small MATH} 165, 166; grades: A, A)
    %% \item Calculus 2 ({\small MATH} 166; grade: A)
    %% \item Calculus 1 ({\small MATH} 165; grade: A)
%%    \end{itemize}



\end{resume}
\end{document}
















%%% Local Variables:
%%% mode: latex
%%% TeX-master: t
%%% End:
